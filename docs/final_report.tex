\documentclass[11pt, a4paper]{article}
\usepackage[utf8]{inputenc}
\usepackage{amsmath, amssymb, amsfonts}
\usepackage{graphicx}
\usepackage{booktabs}
\usepackage{hyperref}
\usepackage{geometry}
\geometry{margin=1in}

\title{\textbf{Market Microstructure Analysis of INFY: A Multivariate Hawkes Process Approach}}
\author{Academic Report}
\date{\today}

\begin{document}

\maketitle

\begin{abstract}
This report presents a comprehensive analysis of high-frequency trading data for INFY (Infosys) on the National Stock Exchange (NSE). We employ a multivariate Hawkes process to model the arrival dynamics of buyer-initiated and seller-initiated trades. Our study combines rigorous data cleaning, aggressor side determination, and maximum likelihood estimation to quantify the degree of endogeneity and self-excitation in the market. The results reveal significant branching ratios, indicating strong clustering of market events. We further discuss the implications of these findings for liquidity profiling and anomaly detection.
\end{abstract}

\section{Introduction}
Understanding the microstructure of financial markets requires disentangling the complex interactions between market participants. The arrival of trades is not a Poisson process but exhibits self-exciting and mutually exciting properties. This study leverages the Hawkes process framework to capture these dynamics for the INFY stock.

\section{Data and Methodology}

\subsection{Data Source and Preprocessing}
The analysis is based on tick-by-tick Order and Trade data for INFY provided by the NSE. 
\begin{itemize}
    \item \textbf{Data Cleaning}: Raw data was parsed using an optimized loader (`NseDataLoaderOptimized`). Timestamps were standardized, and trade records were matched with order records to reconstruct the limit order book context.
    \item \textbf{Sample}: The analysis focuses on a specific trading day (13th August 2019), comprising approximately 85,000 orders and 42,000 trades.
\end{itemize}

\subsection{Aggressor Determination Algorithm}
A critical step in microstructure analysis is identifying the aggressor side—the party that crosses the spread to execute a trade. We implemented a precise algorithm in `models/aggressor\_classifier.py`:
\begin{enumerate}
    \item \textbf{Timestamp Matching}: For every trade, we identified the original entry timestamps of the corresponding Buy and Sell orders from the order book.
    \item \textbf{Logic}: The aggressor is defined as the side with the \textit{later} entry timestamp.
    \begin{equation}
        Aggressor = \begin{cases} 
        +1 (\text{Buyer}) & \text{if } t_{buy}^{entry} > t_{sell}^{entry} \\
        -1 (\text{Seller}) & \text{if } t_{sell}^{entry} > t_{buy}^{entry}
        \end{cases}
    \end{equation}
    \item \textbf{Resting Orders}: If an order's timestamp is missing (implying it is a resting order carried over from the previous day), it is treated as earlier than any new order.
\end{enumerate}

\subsection{Model Specification: Multivariate Hawkes Process}
We model the arrival times of buyer-initiated trades ($N_1$) and seller-initiated trades ($N_2$) using a 2-dimensional Hawkes process. The conditional intensity function $\lambda_i(t)$ for dimension $i \in \{1, 2\}$ is given by:
\begin{equation}
    \lambda_i(t) = \mu_i + \sum_{j=1}^{2} \int_{-\infty}^{t} \phi_{ij}(t-s) dN_j(s)
\end{equation}
where:
\begin{itemize}
    \item $\mu_i$ is the baseline intensity (exogenous rate of events).
    \item $\phi_{ij}(t) = \alpha_{ij} e^{-\beta_{ij} t}$ is the exponential excitation kernel.
    \item $\alpha_{ij}$ represents the immediate impact of an event in dimension $j$ on dimension $i$.
    \item $\beta_{ij}$ dictates the decay rate of the excitation.
\end{itemize}

The \textbf{Branching Ratio} ($n_{ij}$), representing the average number of type $i$ events triggered by a single type $j$ event, is calculated as $n_{ij} = \int_{0}^{\infty} \phi_{ij}(τ) dτ = \frac{\alpha_{ij}}{\beta_{ij}}$.

\section{Empirical Results}

\subsection{Parameter Estimates}
The model was fitted using Maximum Likelihood Estimation (MLE). The estimated parameters are summarized below:

\begin{table}[h]
\centering
\caption{Estimated Hawkes Process Parameters}
\begin{tabular}{lcc}
\toprule
\textbf{Parameter} & \textbf{Buy Node ($i=1$)} & \textbf{Sell Node ($i=2$)} \\
\midrule
Baseline Intensity ($\mu$) & 1.8426 & 1.4801 \\
\midrule
\multicolumn{3}{l}{\textit{Excitation from Buy ($j=1$)}} \\
Alpha ($\alpha_{i1}$) & 0.6976 & 0.1583 \\
Beta ($\beta_{i1}$) & 15.9038 & 8.7844 \\
Branching Ratio ($n_{i1}$) & \textbf{0.0439} & \textbf{0.0180} \\
\midrule
\multicolumn{3}{l}{\textit{Excitation from Sell ($j=2$)}} \\
Alpha ($\alpha_{i2}$) & 0.1389 & 0.6125 \\
Beta ($\beta_{i2}$) & 16.5828 & 13.9119 \\
Branching Ratio ($n_{i2}$) & \textbf{0.0084} & \textbf{0.0440} \\
\bottomrule
\end{tabular}
\end{table}

\subsection{Interpretation}
\begin{itemize}
    \item \textbf{Self-Excitation}: Both buyer and seller trades exhibit self-excitation ($n_{11} \approx 0.04$, $n_{22} \approx 0.04$). This confirms that a trade increases the probability of subsequent trades of the same direction (clustering).
    \item \textbf{Cross-Excitation}: Cross-excitation effects ($n_{12}, n_{21}$) are present but generally weaker than self-excitation, suggesting distinct directional momentum.
    \item \textbf{Stability}: The spectral radius of the branching matrix is well below 1, ensuring the process is stationary and non-explosive.
\end{itemize}

\subsection{Forecasting}
Using the fitted model, we forecasted the expected number of events for a 60-second horizon:
\begin{itemize}
    \item Expected Buy Trades: $\approx 113$
    \item Expected Sell Trades: $\approx 89$
\end{itemize}

\section{Implications}

\subsection{Liquidity Profiling}
The estimated baseline intensities ($\mu$) provide a measure of ``fundamental'' liquidity demand, independent of recent market activity. The branching ratios quantify ``reactive'' liquidity. A higher branching ratio implies that the market is more fragile or sensitive to shocks, as a single order triggers a cascade of follow-up orders. In this case, the moderate branching ratios indicate a relatively stable market for INFY during the observed period.

\subsection{Anomaly Detection}
The Hawkes process framework serves as a robust baseline for anomaly detection. sudden deviations in the observed intensity from the model's predicted conditional intensity $\lambda(t)$ can signal:
\begin{itemize}
    \item \textbf{Flash Crashes}: Rapid cascading of sell orders that violate the stationary branching assumptions.
    \item \textbf{Algorithmic Gaming}: Unusual patterns in cross-excitation (e.g., spoofing) where $\alpha_{12}$ or $\alpha_{21}$ spikes abnormally.
\end{itemize}

\section{Conclusion}
This report successfully demonstrated the application of multivariate Hawkes processes to INFY trade data. The model effectively captures the self-exciting nature of order flows. Future work will extend this analysis to include Order Flow Imbalance (OFI) as a marked point process and explore regime-switching Hawkes models to capture intraday seasonality.

\end{document}
